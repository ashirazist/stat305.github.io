\documentclass[11pt]{article}\usepackage[]{graphicx}\usepackage[]{color}
%% maxwidth is the original width if it is less than linewidth
%% otherwise use linewidth (to make sure the graphics do not exceed the margin)
\makeatletter
\def\maxwidth{ %
  \ifdim\Gin@nat@width>\linewidth
    \linewidth
  \else
    \Gin@nat@width
  \fi
}
\makeatother
\usepackage{pdfpages} 
\definecolor{fgcolor}{rgb}{0.345, 0.345, 0.345}
\newcommand{\hlnum}[1]{\textcolor[rgb]{0.686,0.059,0.569}{#1}}%
\newcommand{\hlstr}[1]{\textcolor[rgb]{0.192,0.494,0.8}{#1}}%
\newcommand{\hlcom}[1]{\textcolor[rgb]{0.678,0.584,0.686}{\textit{#1}}}%
\newcommand{\hlopt}[1]{\textcolor[rgb]{0,0,0}{#1}}%
\newcommand{\hlstd}[1]{\textcolor[rgb]{0.345,0.345,0.345}{#1}}%
\newcommand{\hlkwa}[1]{\textcolor[rgb]{0.161,0.373,0.58}{\textbf{#1}}}%
\newcommand{\hlkwb}[1]{\textcolor[rgb]{0.69,0.353,0.396}{#1}}%
\newcommand{\hlkwc}[1]{\textcolor[rgb]{0.333,0.667,0.333}{#1}}%
\newcommand{\hlkwd}[1]{\textcolor[rgb]{0.737,0.353,0.396}{\textbf{#1}}}%
\let\hlipl\hlkwb

\usepackage{ulem}

\usepackage{framed}
\makeatletter
\newenvironment{kframe}{%
 \def\at@end@of@kframe{}%
 \ifinner\ifhmode%
  \def\at@end@of@kframe{\end{minipage}}%
  \begin{minipage}{\columnwidth}%
 \fi\fi%
 \def\FrameCommand##1{\hskip\@totalleftmargin \hskip-\fboxsep
 \colorbox{shadecolor}{##1}\hskip-\fboxsep
     % There is no \\@totalrightmargin, so:
     \hskip-\linewidth \hskip-\@totalleftmargin \hskip\columnwidth}%
 \MakeFramed {\advance\hsize-\width
   \@totalleftmargin\z@ \linewidth\hsize
   \@setminipage}}%
 {\par\unskip\endMakeFramed%
 \at@end@of@kframe}
\makeatother

\definecolor{shadecolor}{rgb}{.97, .97, .97}
\definecolor{messagecolor}{rgb}{0, 0, 0}
\definecolor{warningcolor}{rgb}{1, 0, 1}
\definecolor{errorcolor}{rgb}{1, 0, 0}
\newenvironment{knitrout}{}{} % an empty environment to be redefined in TeX

\usepackage{alltt}
\usepackage{graphicx, fancyhdr}
\usepackage{amsmath, amsfonts}
\usepackage{color}
\usepackage{hyperref}

\newcommand{\blue}[1]{{\color{blue} #1}}

\setlength{\topmargin}{-.5 in} 
\setlength{\textheight}{9 in}
\setlength{\textwidth}{6.5 in} 
\setlength{\evensidemargin}{0 in}
\setlength{\oddsidemargin}{0 in} 
\setlength{\parindent}{0 in}
\newcommand{\ben}{\begin{enumerate}}
\newcommand{\een}{\end{enumerate}}


\lhead{STAT 305}
\chead{Homework \# 6} 
\rhead{Due Thursday, Oct. $17^{th}$ in the class}
\lfoot{Fall 2019} 
\cfoot{\thepage} 
\rfoot{} 
\renewcommand{\headrulewidth}{0.4pt} 
\renewcommand{\footrulewidth}{0.4pt} 

\def\Exp#1#2{\ensuremath{#1\times 10^{#2}}}
\def\Case#1#2#3#4{\left\{ \begin{tabular}{cc} #1 & #2 \phantom
{\Big|} \\ #3 & #4 \phantom{\Big|} \end{tabular} \right.}
\IfFileExists{upquote.sty}{\usepackage{upquote}}{}
\usepackage{Sweave}
\begin{document}
\Sconcordance{concordance:stat305_hw6.tex:stat305_hw6.Rnw:%
1 83 1 1 0 85 1}

\pagestyle{fancy} 

Show \textbf{all} of your work on this assignment and answer each question fully in the given context. \\


\emph{Please} staple your assignment!

\begin{enumerate}
	

	
	\item Ch. 5.1, Exercise 1, pg. 243 A discrete random variable $X$ can be described using the probability function, $f(x)$:


 \begin{table}[h!]
     \centering
     \begin{tabular}{llllll}
        \hline
         x  & 2  & 3 &  4  & 5 & 6     \\\hline \hline
         f(x) & 0.1 &  0.2 & 0.3  & 0.3 &  0.1  \\\hline \hline

     \end{tabular}
  \end{table}
    
    \begin{enumerate}
          \item  Plot $F(x)$, the cumulative probability function for $X$.[5 pts]
          
          \item Find the mean and standard deviation of $X$.[10 pts]
    \end{enumerate}      
    	
	
	
	\item Ch. 5, Exercise 1, pg. 322: Suppose $90\%$ of all students taking a beginning programming class fail to get their first program to run on first submission. Use a binomial distribution and assign probabilities to the possibilities that among a group of six such students,
	\begin{enumerate}
        \item all fail on their first submissions[5 pts]
        \item  at least four fail on their first submissions[5 pts]
        \item  less than four fail on their first submissions [5 pts] 
    
Continuing to using this binomial model,
        \item what is the mean number who will fail?[5 pts]
        \item what are the variance and standard deviation of the number who will fail?[5 pts]
  \end{enumerate}


% \item Let $X$ be a random variable with the probability function given by
% $$f(x) = (.3)^x \cdot (.7)^{1-x}$$
% for $x = 0, 1$ and 0 otherwise.\\
% Find the expected value and variance of $X$.[10 pts]

\item Ch. 5, Exercise 2, pg. 322: Suppose that for single launches of a space shuttle, there is a constant probability of O-ring failure (say $.15$), Consider ten future launches, and let $X$ be the number of those involving an O-ring failure. Use an appropriate probability model and evaluate all of the following:
    \begin{enumerate}
          \item Precisely state the distribution of $X$, giving the values of any parameters necessary.[5 pts]
          \item $P[X = 2]$[5 pts]
          \item $P[X \ge 1]$[5 pts]
          \item $\text{E}X$[5 pts]
          \item $\text{Var} X$[5 pts]
          \item the standard deviation of $X$[5 pts]
    \end{enumerate}

\item Ch. 5.1, Exercise 6, pg. 244: Suppose that an eddy current nondestructive evaluation technique for identifying cracks in critical metal parts has a probability of about $.20$ of detecting a single crack of length $.003$in. in a certain material. Let $Y$ be the number of specimens inspected in order to obtain the first crack detection. Use an appropriate probability model and evaluate all of the following:
    \begin{enumerate}
        \item Precisely state the distribution of $X$, giving the values of any parameters necessary.[5 pts]
        \item $P[Y = 5]$[5 pts]
        \item $P[Y \le 4]$[5 pts]
        \item $\text{E}Y$[5 pts]
        \item $\text{Var}Y$[5 pts]
        \item $\text{SD}(Y)$[5 pts]
    \end{enumerate}


\end{enumerate}


Total: 100 pts









\end{document}
