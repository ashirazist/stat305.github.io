\documentclass[11pt]{article}
\usepackage{graphicx, fancyhdr}
\usepackage{amsmath, amsfonts}
\usepackage{color, hyperref}
\usepackage{enumerate}
\providecommand{\tightlist}{%
  \setlength{\itemsep}{0pt}\setlength{\parskip}{0pt}}
 
\newcommand{\blue}[1]{{\color{blue} #1}}

\setlength{\topmargin}{-.375 in}
\setlength{\textheight}{8.75 in}
\setlength{\textwidth}{6.5 in}
\setlength{\evensidemargin}{0 in}
\setlength{\oddsidemargin}{0 in}
\setlength{\parindent}{0 in}
\renewcommand{\headrulewidth}{0.4pt}
\renewcommand{\footrulewidth}{0.4pt}

\lhead{Stat 305} 
\chead{Homework 10} 
\rhead{Wednesday, May 8}
\lfoot{Spring 2019}
\cfoot{\thepage} 
\rfoot{} 

\def\Exp#1#2{\ensuremath{#1\times 10^{#2}}}
\def\Case#1#2#3#4{\left\{ \begin{tabular}{cc} #1 & #2 \phantom
{\Big|} \\ #3 & #4 \phantom{\Big|} \end{tabular} \right.}

\begin{document}
\pagestyle{fancy} 

Show \textbf{all} of your work on this assignment and answer each question fully in the given context. \\

\emph{Please} staple your assignment!! \\

\begin{enumerate}
\def\labelenumi{\arabic{enumi}.}
\item
  The ``treadwear warranty'' of a tire is often used as a tool to
  communicate to consumers some idea of the tire's treadlife. Assuming
  that the conditions of the warranty are met (for instance, that the
  tires are regularly rotated), the warranty may allow the tire
  purchaser to be reimbursed for lost mileage if the tire does not last
  as long as the warranty indicated. For instance, if an \$80.00 tire
  has a tread wear warranty of 60,000 miles and is sufficiently worn
  down at 45,000 miles, the manufacturer may be required to pay for the
  remaining 15,000 - in other words, to reimburse the customer \$20.00.
  In light of this, consider the following scenario:

  An engineer working for a tire company has developed a very cheap tire
  - the cost of production is \$20.00. 40 of these tires are sampled in
  order to determine the lifetime of the tire in terms of miles it it
  can be driven until the tread depth is 2/32 of an inch (the legal
  minimum tread depth). The sampled tire lifetimes are recorded below
  (in thousands of miles):

  \begin{quote}
  41.3, 30.7, 39.7, 31.8, 39.3, 35, 33.6, 38.9, 31.7, 38.1, 43.9, 30.6,
  33.3, 29.8, 35.9, 34.8, 34.6, 36, 41.5, 37.2, 40.4, 38.4, 36.8, 32.5,
  27.7, 32.1, 35.3, 36.7, 39.4, 40.6, 31.2, 33.9, 33, 43.9, 33.6, 32.6,
  36.2, 32.5, 31.7, 31.5
  \end{quote}

  The average lifetime of the 40 tires is \(\bar{x} = 35.44\) thousand
  miles and the sample variance is \(s^2 = 15.88\) thousand miles
  squared.

  \begin{enumerate}
  \def\labelenumii{\alph{enumii}.}
  \item
    Provide a 95\% confidence lower bound for the mean lifetime of this
    type tire.
  \item
    Suppose market research suggests the tire could be sold with a
    60,000 mile treadwear warranty for \$60.00. If the company
    reimburses mileage \$1.00 for every thousand miles short of 60,000
    the tire travels, is there evidence they could make money on this
    tire?
  \end{enumerate}
\item
  A common rule-of-thumb for safe following distance is to stay two
  seconds behind the lead car. For a car travelling at 60 mph, a safe
  stopping distance would thus be \(176\) feet.

  In order to test this requirement, a common current model of car is
  accelerated to 60 mph, after which point a signal is given for the
  driver to bring the car to a complete halt. The distance required from
  the point the driver was signaled to the point the car came to a halt
  is recorded. The experiment is repeated 10 times, and the resulting
  stopping distances are recorded below:

  \begin{quote}
  175, 187.5, 200.55, 180.4, 200.12, 186.77, 188.9, 183.78, 175.16,
  183.86
  \end{quote}

  \begin{enumerate}
  \def\labelenumii{\alph{enumii}.}
  \item
    Find the sample mean, \(\bar{x}\), and the sample standard
    deviation, \(s^2\), for the sample.
  \item
    For \(\mu\) representing the true mean stopping time, conduct a 95\%
    confidence test for the null hypothesis \(\mu \le 176 \text{ feet}\)
    against the alternative hypothesis \(\mu > 176 \text{ feet}\).
    Include the hypothesis statement, your test statistic, an estimate
    of the p-value, and your conclusion.
  \end{enumerate}
\end{enumerate}

\newpage

\begin{enumerate}
\def\labelenumi{\arabic{enumi}.}
\setcounter{enumi}{2}
\item
  While ornate stained glass doors can improve the curb appeal of a new
  construction home, the process of installing such doors can be a
  headache for the construction companies building them. There are
  several stages where the process can go wrong - the door can be
  damaged at the warehouse, while being shipped, while waiting to be
  installed, by construction mishaps after installation, etc. A
  construction company looking to minimize the problems associated with
  reordering such doors is attempting to determine what proportion of
  doors actually arrive on the job site damaged. The company's current
  working belief is that 10\% of fragile products will be damaged during
  shipping. Keeping track of the next 50 doors shipped, they found 4 of
  the doors arrived already damaged.

  Suppose that we are interested in \(p\), the probability that a door
  is damaged during shipping. In order to do inference on \(p\), we need
  to find a test statistic that connects our observable data (the number
  of damaged doors) to \(p\) through a distribution. Assuming the
  shipments are independent, the following problems help us find such a
  test statistic:

  If we define a Bernoulli random variable, \(X_i\), for each shipment
  as \(X_i = 1\) if Door \(i\) is damaged and \(X_i\) = 0 if Door \(i\)
  is undamaged, then we can also say the following \(P(X_i = 1) = p\)
  (which is just the probability a door is damaged during shippinig).

  \begin{enumerate}
  \def\labelenumii{\alph{enumii}.}
  \tightlist
  \item
    Find \(E(X_i)\)
  \item
    Find \(Var(X_i)\)
  \end{enumerate}

  Define the sample mean of the \(X_i\) random variables as
  \(\bar{X} = \frac{1}{n} \sum_{i=1}^n X_i\).

  \begin{enumerate}
  \def\labelenumii{\alph{enumii}.}
  \setcounter{enumii}{2}
  \tightlist
  \item
    Find \(E(\bar{X})\)
  \item
    Find \(Var{\bar{X}}\)
  \item
    What is the distribution of \(\bar{X}\)?
  \end{enumerate}

  At this point we should be able to recognize a test statistic we can
  use to test hypotheses.

  \begin{enumerate}
  \def\labelenumii{\alph{enumii}.}
  \setcounter{enumii}{5}
  \tightlist
  \item
    Perform a hypothesis test for the claim that the probability of a
    door being damaged is less than 0.10. Include the hypothesis
    statement, your test statistic, an estimate of the p-value, and your
    conclusion.
  \end{enumerate}
\end{enumerate}

\end{document}
