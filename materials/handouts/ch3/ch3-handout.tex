\documentclass[11pt]{article}\usepackage[]{graphicx}\usepackage[]{color}
%% maxwidth is the original width if it is less than linewidth
%% otherwise use linewidth (to make sure the graphics do not exceed the margin)
\makeatletter
\def\maxwidth{ %
	\ifdim\Gin@nat@width>\linewidth
	\linewidth
	\else
	\Gin@nat@width
	\fi
}
\makeatother
\usepackage{pdfpages} 
\definecolor{fgcolor}{rgb}{0.345, 0.345, 0.345}
\newcommand{\hlnum}[1]{\textcolor[rgb]{0.686,0.059,0.569}{#1}}%
\newcommand{\hlstr}[1]{\textcolor[rgb]{0.192,0.494,0.8}{#1}}%
\newcommand{\hlcom}[1]{\textcolor[rgb]{0.678,0.584,0.686}{\textit{#1}}}%
\newcommand{\hlopt}[1]{\textcolor[rgb]{0,0,0}{#1}}%
\newcommand{\hlstd}[1]{\textcolor[rgb]{0.345,0.345,0.345}{#1}}%
\newcommand{\hlkwa}[1]{\textcolor[rgb]{0.161,0.373,0.58}{\textbf{#1}}}%
\newcommand{\hlkwb}[1]{\textcolor[rgb]{0.69,0.353,0.396}{#1}}%
\newcommand{\hlkwc}[1]{\textcolor[rgb]{0.333,0.667,0.333}{#1}}%
\newcommand{\hlkwd}[1]{\textcolor[rgb]{0.737,0.353,0.396}{\textbf{#1}}}%
\let\hlipl\hlkwb

\usepackage{ulem}

\usepackage{framed}
\makeatletter
\newenvironment{kframe}{%
	\def\at@end@of@kframe{}%
	\ifinner\ifhmode%
	\def\at@end@of@kframe{\end{minipage}}%
\begin{minipage}{\columnwidth}%
	\fi\fi%
	\def\FrameCommand##1{\hskip\@totalleftmargin \hskip-\fboxsep
		\colorbox{shadecolor}{##1}\hskip-\fboxsep
		% There is no \\@totalrightmargin, so:
		\hskip-\linewidth \hskip-\@totalleftmargin \hskip\columnwidth}%
	\MakeFramed {\advance\hsize-\width
		\@totalleftmargin\z@ \linewidth\hsize
		\@setminipage}}%
{\par\unskip\endMakeFramed%
	\at@end@of@kframe}
\makeatother

\definecolor{shadecolor}{rgb}{.97, .97, .97}
\definecolor{messagecolor}{rgb}{0, 0, 0}
\definecolor{warningcolor}{rgb}{1, 0, 1}
\definecolor{errorcolor}{rgb}{1, 0, 0}
\newenvironment{knitrout}{}{} % an empty environment to be redefined in TeX

\usepackage{alltt}
\usepackage{graphicx, fancyhdr}
\usepackage{amsmath, amsfonts}
\usepackage{color}
\usepackage{hyperref}

\newcommand{\blue}[1]{{\color{blue} #1}}

\setlength{\topmargin}{-.5 in} 
\setlength{\textheight}{9 in}
\setlength{\textwidth}{6.5 in} 
\setlength{\evensidemargin}{0 in}
\setlength{\oddsidemargin}{0 in} 
\setlength{\parindent}{0 in}
\newcommand{\ben}{\begin{enumerate}}
	\newcommand{\een}{\end{enumerate}}


\lhead{STAT 305}
\chead{Handout Ch 3} 
\rhead{September $10^{th}$}
\lfoot{Fall 2019} 
\cfoot{\thepage} 
\rfoot{} 
\renewcommand{\headrulewidth}{0.4pt} 
\renewcommand{\footrulewidth}{0.4pt} 

\def\Exp#1#2{\ensuremath{#1\times 10^{#2}}}
\def\Case#1#2#3#4{\left\{ \begin{tabular}{cc} #1 & #2 \phantom
		{\Big|} \\ #3 & #4 \phantom{\Big|} \end{tabular} \right.}
\IfFileExists{upquote.sty}{\usepackage{upquote}}{}
\begin{document}
	\pagestyle{fancy} 
	
The following data were gathered for an analysis on Manganese in some engineering system. Read the data and answer the questions based on the data. 
\begin{center}
			\begin{tabular}{cccc}
			74& 79& 77& 81\\         
		    68& 79& 81& 76\\         
		    81& 80& 80& 78\\         
		    88& 83& 79& 91\\         
		    79& 75& 74& 73\\ 

			\end{tabular}
\end{center}

\begin{itemize}
		
		\item \textbf{Draw the stem and leaf diagram:} \\
\vspace{2 in}
		\item \textbf{ Make a frequency table }			
\begin{center}
	\begin{tabular}{p{2cm}| p{2cm}|p{3cm}|p{2cm}|p{2cm}|p{2cm}}
		Class & Interval Center& Tally & Frequency & Relative Frequency & Cumulative r.f \\
		\hline
		& & & & & \vspace{0.3cm}  \\
		  
		\hline      
		& & & & &\vspace{0.3cm} \\        
		\hline
		& & & &  &\vspace{0.3cm}\\         
		\hline
		& & & & &\vspace{0.3cm} \\         
		\hline
		& & & & &\vspace{0.3cm} \\ 
				\hline
		& & & & &\vspace{0.3cm} \\ 
	\end{tabular}
\end{center}

		\item \textbf{Draw the Histogram of the data}
		\pagebreak
		
		\item \textbf{Calculate the quantiles }
\begin{center}
	\begin{tabular}{p{1cm} p{1cm}p{1cm}p{1cm}p{1cm}p{1cm} p{1cm}p{1cm}p{1cm}p{1cm}p{1cm}}
		i &1 & 2 &  3&  4& 5 & 6  &7  &8 &9 &10  \\
		\hline
		Data& 68 &73& 74& 74& 75& 76& 77& 78& 79& 79\\
		\hline
		\vspace{.15cm}
		$\frac{i- 0.5}{n}$\vspace{.15cm}\\
		\hline
		\vspace{.15cm}
		Q(p)\vspace{.15cm}\\
		\hline
				\vspace{.15cm}
		$Q_N(p)$\vspace{.15cm}\\
		\hline
		
	\end{tabular}
\end{center}		

\begin{center}
	\begin{tabular}{p{1cm} p{1cm}p{1cm}p{1cm}p{1cm}p{1cm} p{1cm}p{1cm}p{1cm}p{1cm}p{1cm}}
		i &11 & 12 &  13&  14& 15 & 16  &17  &18 &19 &20  \\
		\hline
		Data& 79& 79& 80& 80& 81& 81& 81& 83& 88& 91\\
		\hline
				\vspace{.15cm}
		$\frac{i- 0.5}{n}$\vspace{.15cm}\\
		\hline
		\vspace{.15cm}
		Q(p)\vspace{.15cm}\\
		\hline
		\vspace{.15cm}
		$Q_N(p)$\vspace{.15cm}\\
		\hline

	\end{tabular}
\end{center}			
	 	
	 	\item \textbf{Find the median, $1^{st}$ quartile and   $3^{rd}$ quartile }	\\

		

\vspace{1.5cm}
	
		\item \textbf{Find the Normal quantiles and add them to the quantile table above}
		
		\item \textbf{Plot the Normal quantiles vs. the data quantiles}
		\vspace{7cm}
		\item \textbf{Draw the boxplot}
		
		
		
		
			\end{itemize}
\end{document}
