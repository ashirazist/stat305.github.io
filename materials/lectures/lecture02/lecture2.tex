\documentclass[]{article}
\usepackage{lmodern}
\usepackage{amssymb,amsmath}
\usepackage{ifxetex,ifluatex}
\usepackage{fixltx2e} % provides \textsubscript
\ifnum 0\ifxetex 1\fi\ifluatex 1\fi=0 % if pdftex
  \usepackage[T1]{fontenc}
  \usepackage[utf8]{inputenc}
\else % if luatex or xelatex
  \ifxetex
    \usepackage{mathspec}
  \else
    \usepackage{fontspec}
  \fi
  \defaultfontfeatures{Ligatures=TeX,Scale=MatchLowercase}
\fi
% use upquote if available, for straight quotes in verbatim environments
\IfFileExists{upquote.sty}{\usepackage{upquote}}{}
% use microtype if available
\IfFileExists{microtype.sty}{%
\usepackage{microtype}
\UseMicrotypeSet[protrusion]{basicmath} % disable protrusion for tt fonts
}{}
\usepackage[margin=1in]{geometry}
\usepackage{hyperref}
\hypersetup{unicode=true,
            pdfborder={0 0 0},
            breaklinks=true}
\urlstyle{same}  % don't use monospace font for urls
\usepackage{graphicx,grffile}
\makeatletter
\def\maxwidth{\ifdim\Gin@nat@width>\linewidth\linewidth\else\Gin@nat@width\fi}
\def\maxheight{\ifdim\Gin@nat@height>\textheight\textheight\else\Gin@nat@height\fi}
\makeatother
% Scale images if necessary, so that they will not overflow the page
% margins by default, and it is still possible to overwrite the defaults
% using explicit options in \includegraphics[width, height, ...]{}
\setkeys{Gin}{width=\maxwidth,height=\maxheight,keepaspectratio}
\IfFileExists{parskip.sty}{%
\usepackage{parskip}
}{% else
\setlength{\parindent}{0pt}
\setlength{\parskip}{6pt plus 2pt minus 1pt}
}
\setlength{\emergencystretch}{3em}  % prevent overfull lines
\providecommand{\tightlist}{%
  \setlength{\itemsep}{0pt}\setlength{\parskip}{0pt}}
\setcounter{secnumdepth}{0}
% Redefines (sub)paragraphs to behave more like sections
\ifx\paragraph\undefined\else
\let\oldparagraph\paragraph
\renewcommand{\paragraph}[1]{\oldparagraph{#1}\mbox{}}
\fi
\ifx\subparagraph\undefined\else
\let\oldsubparagraph\subparagraph
\renewcommand{\subparagraph}[1]{\oldsubparagraph{#1}\mbox{}}
\fi

%%% Use protect on footnotes to avoid problems with footnotes in titles
\let\rmarkdownfootnote\footnote%
\def\footnote{\protect\rmarkdownfootnote}

%%% Change title format to be more compact
\usepackage{titling}

% Create subtitle command for use in maketitle
\providecommand{\subtitle}[1]{
  \posttitle{
    \begin{center}\large#1\end{center}
    }
}

\setlength{\droptitle}{-2em}

  \title{}
    \pretitle{\vspace{\droptitle}}
  \posttitle{}
    \author{}
    \preauthor{}\postauthor{}
    \date{}
    \predate{}\postdate{}
  

\begin{document}

\section{STAT 305: Lecture 2}\label{stat-305-lecture-2}

\subsection{Why Engineers Study
Statistics}\label{why-engineers-study-statistics}

\subsubsection{Chapter 1: Introduction,
Continued}\label{chapter-1-introduction-continued}

\subsubsection{Chapter 2: Data
Collection}\label{chapter-2-data-collection}

\subsection{\texorpdfstring{.footnote{[}Course page:
\href{https://ashirazist.github.io/stat305.github.io/}{ashirazist.github.io/stat305.github.io}{]}}{.footnote{[}Course page: ashirazist.github.io/stat305.github.io{]}}}\label{footnotecourse-page-ashirazist.github.iostat305.github.io}

\section{Section 1.2}\label{section-1.2}

\subsection{\#\# Basic Terminology,
Continued}\label{basic-terminology-continued}

layout:false .left-column{[} \#\# What and Why \#\# Terms \#\# Data
Structures{]} .right-column{[} \#\# Types of Data Structures

The most basic way to think about data is to imagine how the the raw
observations could be organized once collected.

Collected data can be referred to as a \textbf{data set}. If the data
set is simple enough, we can store it in a \textbf{data table} or
\textbf{flat file}. Traditional data tables store values relating to a
single observation/unit/individual as a row of the table. Each column in
the table represents a value for some observed characterstic observed.

\textbf{Example}: Failure time of lightbulbs

A single brand and model of lightbulb is being examined for average
failure time. Five bulbs were run until they burned out and their
lifetime was recorded in hours. The first bult lasted 521.4 hours, the
second bulb lasted 501.2 hours, the third bulb lasted 541.8 hours, the
fourth bulb lasted 498.1 hours, and the fifth bulb lasted 528.2 hours.
{]} --- layout:false .left-column{[} \#\# What and Why \#\# Terms \#\#
Data Structures{]} .right-column{[} \#\# Types of Data Structures

\textbf{Example}: Failure time of lightbulbs, continued

Assembling the results in a data table could look like this:

\begin{verbatim}
Bulb Number       Failure Time (hours)
1                 521.4
2                 501.2
3                 541.8
4                 498.1
5                 528.2
\end{verbatim}

Each bulb tested gets its own row - which row is attached to which bulb
is identified by the first column. The only feature being observed is
failure time - so only one column of observations are recorded for each
bulb.

Notice:

\begin{itemize}
\tightlist
\item
  Failure Time is a \textbf{quantitative continuous} variable.
\item
  This is a \textbf{univariate data set}.
\end{itemize}

\subsection{{]}}\label{section}

layout:false .left-column{[} \#\# What and Why \#\# Terms \#\# Data
Structures{]} .right-column{[} \#\# Types of Data Structures

\textbf{Example}: Type of bill, date of payment, and payment amount for
Mediacom

\begin{verbatim}
Customer      Type      Date          Amount
John Doe      Internet  01-05-2015    110.00
John Doe      Phone     01-15-2015     10.00
John Doe      Internet  02-05-2015    110.00
John Doe      Phone     02-15-2015     10.00
John Doe      Internet  03-05-2015    110.00
John Doe      Phone     03-15-2015     10.00
...           ...       ...           ...
John Doe      Internet  01-05-2016    110.00
John Doe      Phone     01-15-2016     10.00
Jane Doe      Internet  04-12-2015     90.00
Jane Doe      Internet  05-12-2015     90.00
...           ...       ...           ...
Jane Doe      Internet  01-12-2016     90.00
\end{verbatim}

Notice:

\begin{itemize}
\tightlist
\item
  Type of bill is is a \textbf{Qualitative} variable.
\item
  Amount paid is \textbf{quantitative discrete}.
\item
  Date is \ldots{} {]} --- layout:false .left-column{[} \#\# What and
  Why \#\# Terms \#\# Data Structures{]} .right-column{[} \#\# Types of
  Data Structures
\end{itemize}

\textbf{Example}: Machine Parts \textgreater{} Suppose we get a shipment
of 5000 machine parts and would like to verify that the shipment meets
the standards the machinist agreed to. We take out 100 parts and examine
them carefully. To verify that the parts are as strong as we
anticipated, we measure the ``Rockwell hardness'' with a machine that is
accurate to the first decimal place. We also examine each part for
scratches and record it weight. Further, we run the part in a test
machine to determine if it works correctly.

In this case, we are gathering \textbf{4} values on each part. So for
instance, the first of the 100 parts we examine could have a measured
Rockwell hardness of 3.2, no scratches, a weight of 1.7562 g, and it
works correctly. The second of the 100 parts we examine could have a
measured Rockwell hardness of 3.1, no scratches, a weight of 1.7901 g,
and does not work correctly. {]} --- layout:false .left-column{[} \#\#
What and Why \#\# Terms \#\# Data Structures{]} .right-column{[} \#\#
Types of Data Structures

The data as recorded by the researcher might look like this

\begin{verbatim}
Part identifier: 1/100
  Rockwell Hardness: 3.2
  scratches: no
  weight (g): 1.7562
  functioning: yes

Part identifier: 2/100
  Rockwell Hardness: 3.1
  scratches: no
  weight (g): 1.7901
  functioning: no

...

Part identifier: 100/100
  Rockwell Hardness: 3.4
  scratches: no
  weight (g): 1.7651
  functioning: yes
\end{verbatim}

\subsection{{]}}\label{section-1}

layout:false .left-column{[} \#\# What and Why \#\# Terms \#\# Data
Structures{]} .right-column{[} \#\# Types of Data Structures

Which we could turn into structured data table like this: The data as
recorded by the researcher might look like this

\begin{verbatim}
part rockwell_hardness   weight scratches functioning
1                  3.2   1.7562        no         yes
2                  3.1   1.7901        no          no
.                    .        .         .           .
.                    .        .         .           .
.                    .        .         .           .
100                3.4   1.7651        no         yes
\end{verbatim}

When data is arranged like this, with each sampling unit on its own row,
the data is said to be in \textbf{wide format}. {]} --- layout:false
.left-column{[} \#\# What and Why \#\# Terms \#\# Data Structures{]}
.right-column{[} \#\# Types of Data Structures

However, we could also structure a data table like this:

\begin{verbatim}
part    measurement    value
1       Rockwell         3.2
1       weight        1.7562
1       scratches         no
1       functioning      yes
2       Rockwell         3.1
2       weight        1.7901
2       scratches         no
2       functioning       no
.       .                  .
.       .                  .
.       .                  .
100     functioning      yes
\end{verbatim}

When data is arranged like this, with each sampling unit on its own row,
the data is said to be in \textbf{long format}. Long format matches each
recorded value to a unique set of identifiers called \textbf{keys} - in
this case, for example, the first row matches the recorded value 3.2
uniquely to the measurement Rockwell hardness and the first part in our
sample. {]}

layout:false .left-column{[} \#\# What and Why \#\# Terms \#\# Data
Structures{]} .right-column{[} \#\# Factorial Studies

\begin{quote}
\textbf{Factorial Studies} involve scenarios in which several process
variables are indentified as being of interest and data are collected
under different settings of these process variables.
\end{quote}

\begin{quote}
We call the process variables \textbf{factors} and the possible settings
for a process variable its \textbf{levels}
\end{quote}

\begin{quote}
\textbf{Complete Factorial Studies} are factorial studies where data is
collected from each possible combination of the levels of the factors.
\end{quote}

\begin{quote}
\textbf{Partial Factorial Studies} are factorial studies where data is
collected from some (but not all) possible combinations of the levels of
the factors.
\end{quote}

\subsection{{]}}\label{section-2}

layout:false .left-column{[} \#\# What and Why \#\# Terms \#\# Data
Structures{]} .right-column{[}

\subsection{Factorial Studies Example}\label{factorial-studies-example}

\begin{quote}
A pair of chemists, Walter and Jessie, are attempting to synthesize a
chemical product and consider purity to be the most important quality.
There are three environments available to them (a winnebago, a basement,
and a laboratory) and two precursors (pseudoephedrine/methylamine). They
are both willing to take the role of ``lead cook'' and will try all
their options in order to get the best results.
\end{quote}

\begin{itemize}
\item
  What parts of this synthesis are being treated as variables which can
  be controlled at the start of the experiment?
\item
  What are the possible values for each of these variables?
\item
  How many ways can the variables be combined? {]}
\end{itemize}

???

lead cook - Walter, Jessie environment - winnebago, basement, lab
precursor - pseudo, methylamine

2 x 3 x 2 = 12

\subsection{WRITE OUT THE LEVELS AFTER SOME TIME FOR THEM TO WRITE THEM
OUT}\label{write-out-the-levels-after-some-time-for-them-to-write-them-out}

layout:false .left-column{[} \#\# What and Why \#\# Terms \#\# Data
Structures{]} .right-column{[}

Factorial Studies Example, cont

Here are all the possible combinations of the factors:

\[
\scriptsize{
(\text{# of Cooks}) \cdot  
(\text{# of Environments}) \cdot  
(\text{# of Precursors}) = 2 \cdot 3 \cdot 2  
= 12}
\]

\begin{verbatim}
         cook     environment    precursor
         walter   winnebago      psuedoephedrine
         walter   winnebago      methylamine
         walter   basement       psuedoephedrine
         walter   basement       methylamine
         walter   lab            psuedoephedrine
         walter   lab            methylamine
         jessie   winnebago      psuedoephedrine
         jessie   winnebago      methylamine
         jessie   basement       psuedoephedrine
         jessie   basement       methylamine
         jessie   lab            psuedoephedrine
         jessie   lab            methylamine
\end{verbatim}

If we collect data from each of these combinations, we have performed a
\textbf{A Complete Factorial Study} {]}

\subsection{???}\label{section-3}

layout:false .left-column{[} \#\# What and Why \#\# Terms \#\# Data
Structures{]} .right-column{[}

Factorial Studies Example, cont

After testing each scenario, Walter and Jessie decide that the best
combination to use is Walt as cook in the lab with methylamine. However,
a new ``chemist'' Victor has joined the group and is going to try to be
the cook and ``follow the recipe'' in the lab. Jessie also tries a new
environment, South America, where only methylamine is available.

\begin{itemize}
\item
  If we consider the all the past combinations to be part of this new
  study, how many combinations of factor levels are now possible?
\item
  Victor never works in the Winnebago, the basement, or South America.
  Walter never works in South America. {]} --- layout:false
  .left-column{[} \#\# What and Why \#\# Terms \#\# Data Structures{]}
  .right-column{[}
\end{itemize}

Factorial Studies Example, cont

\begin{verbatim}
     cook     env         precursor
1.   walt     winne       pseudo
2.   walt     winne       methylamine
3.   walt     basement    pseudo
4.   walt     basement    methylamine
5.   walt     lab         pseudo
6.   walt     lab         methylamine
7.   jessie   winne       pseudo
8.   jessie   winne       methylamine
9.   jessie   basement    pseudo
10.  jessie   basement    methylamine
11.  jessie   lab         pseudo
12.  jessie   lab         methylamine
13.  jessie   so. am.     methylamine
14.  victor   lab         methylamine
\end{verbatim}

In this case, we would have a \textbf{Fractional Factorial Study} - a
factorial study in which no data is collected for some possible
combinations. {]}

???

\section{Section 1.3}\label{section-1.3}

\subsection{\#\# Measurement: It's Importance and
Difficulty}\label{measurement-its-importance-and-difficulty}

layout:false .left-column{[} \#\# What and Why \#\# Terms \#\# Measure
\#\#\# Key Words{]} .right-column{[}

\subsection{If You Can't Measure, You Can't Do
Statistics}\label{if-you-cant-measure-you-cant-do-statistics}

\subsubsection{Or Engineering For That
Matter}\label{or-engineering-for-that-matter}

\begin{itemize}
\tightlist
\item
  \textbf{Validity}: faithfully representing the aspect of interest
\item
  \textbf{Precision}: the amount of variation in repeated measures
\item
  \textbf{Accuracy}: aka ``unbiasedness''; how close a measurement is to
  the true value ``on average''
\end{itemize}

We \textbf{calibrate} to improve accuracy {]}

???

\section{Section 1.4}\label{section-1.4}

\subsection{\#\# Mathematical Models}\label{mathematical-models}

layout:false .left-column{[} \#\# What and Why \#\# Terms \#\# Measure
\#\# Math Models{]} .right-column{[}

\subsection{Mathematical Models and Data
Analysis}\label{mathematical-models-and-data-analysis}

\begin{quote}
\textbf{Mathematical Model}: A description of a physical system using
mathematical concepts and language.
\end{quote}

Identifying mathematical relationships between parts of a system allows
us to describe complexity in simple terms.

\textbf{Example}: Height of an Object in Projectile Motion

We can describe the relationship between height of a projectile
\textbackslash{}(y\textbackslash{}) and time
\textbackslash{}(t\textbackslash{}) as \textbackslash{}{[} y = h\_0 +
v\_h \cdot t - \frac{1}{2} g t\^{}2, \text{ } t \ge 0,
\textbackslash{}{]} where - \textbackslash{}(h\_0\textbackslash{}) is
the initial height, - \textbackslash{}(v\_h\textbackslash{}) is the
initial vertical velocity, and - \textbackslash{}(g\textbackslash{}) is
the (constant) acceleration due to gravity {]}

???

\begin{center}\rule{0.5\linewidth}{\linethickness}\end{center}

layout:false .left-column{[} \#\# What and Why \#\# Terms \#\# Measure
\#\# Math Models{]} .right-column{[}

\textbf{Example}: Height of an Object in Projectile Motion, cont.

\textbackslash{}{[} y = h\_0 + v\_h \cdot t - \frac{1}{2} g t\^{}2,
\text{   } t \ge 0, \textbackslash{}{]}

However, this is not what we see in real life for a variety of reasons.
This model assumes

\begin{enumerate}
\def\labelenumi{\arabic{enumi}.}
\item
  \textbackslash{}(g\textbackslash{}) is constant as the ball falls,
  while \textbackslash{}(g\textbackslash{}) actually depends on the
  distance between the object and earth,
\item
  \textbackslash{}(g\textbackslash{}) is a known to infinite accuracy,
  while we would be using a value that is estimated,
\item
  Gravity is the only force acting on the object, ignoring drag force,
  electrical attractions, etc.
\item
  There are no other changes in the system (for instance, changes in air
  pressure)
\end{enumerate}

We can fix these by writing a better relationship \emph{or} we can
accept that some things won't be known and use a \textbf{stochastic
model} - a mathematical model that specifically allows for variation (or
``randomness''). Understanding how these \textbf{stochastic models} work
is a major focus of this course. {]}


\end{document}
